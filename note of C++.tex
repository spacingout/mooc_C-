\documentclass[a4paper,12pt]{article}

\usepackage{CJK,indentfirst}%indenfirst宏包允许设置首行缩进

\begin{document}

\begin{CJK*}{UTF8}{gbsn}%用宋体

 

\part{启航~ }

\begin{flushleft}
1、新的数据类型:bool.\\
2、初始化方法\\
(1)复制初始化\\
(2)直接初始化 int x(1024)\\
3、随用随定义\\
4、cout\verb|<<|oct\verb|<<|x\verb|<<|endl;可以输出八进制的数,dec,hex同理\\
4、cout\verb|<<|boolalpha\verb|<<|y\verb|<<|endl;输出bool类型。\\
5、命名空间   namespace\\
    (1)using namespace A –using namespace B\\
         分别说明命名空间,如果冲突,必须说明,否则不必说明\\
         用的次数较少:B::fun();(用一次写一次)\\
         用的次数多:cout\verb|<<|A::x\verb|<<|endl;(在前面直接加namespace std)\\
\end{flushleft}
\part{离港篇}
 

   
\section{引用}
\begin{flushleft} 
**就是起外号~\\

1、对用外号进行操作和直接对变量本身进行操作效果是一样的\\
 (1)基本数据类型的引用:\\
\qquad int a=3;\\
\qquad int \&b=a   //引用必须初始化\\
\qquad b=10\\
\qquad cout\verb|<<|a\verb|<<|endl;  此时输出的结果将为10\\

 (2)指针类型的引用:\\
\qquad int a=10;\\
\qquad int *q=\&a;\\
\qquad int \verb|*&|q=p;   就是比基本数据类型多了个星号\\
\qquad *q=20;\\
(3)引用作函数参数:\\
int x=10,y=30;\\
fun(x,y)     --这里直接写主函数里使用的目标变量即可\\
void fun(int \&a,int \&b)\\
\{\\
\qquad   int c=0;\\
\qquad   c=a;\\
\qquad   a=b;\\
\qquad   b=c;\\
\}\\
其原理就是“打萝卜头=打罗小明”;\\
\end{flushleft}

\section{const}
\begin{flushleft} 
宏定义define出来的x不检查语法错误,而const定义的x是有类型的是要检查语法错误的,所以更推荐使用const\\
1、const与基本数据类型\\
\quad const int x=3; 表示此时x无法再通过赋值来改变值\\
2、const与指针类型\\
(1)几种写法:\\
\quad const int *p=NULL;\\
\quad int const *p=NULL;\\
\quad int *const p=NULL;\\
前两种完全等价,但和第三种不同;\\
\quad const int *const p=NULL;\\
\quad int const *const p=NULL;\\
以上两种也是完全等价的写法。\\
 (2)典型错误:\\
①  int x=3; const int *p=\& x;\\
 p=\& y;  \verb|---->|正确\\
 *p=4;   \verb|---->|错误\\ 因为之前const修饰的是*p,所以p指向的值不能变,但可以指向别的\\
②  int x=2;int *const p=\& x;p=\& y;将是错误的;\\
因为此时const修饰的p,即不能改变指向\\
③  const int x=3;const int *const p=\&x \\
    p=\& y; *p=4; 都将是错误的  \\
    因为有俩const,所以值和指向都不能变。\\
④  const int x=3; int *y=\& x; \verb|-->|错误\\
    不变的x权限小,变的y权限大,容易出事,所以编译器直接不给过\\
    int x=3;  const int *y=\& x;\verb|-->|正确\\
    变的x权限大,不变的y权限小\\
3、const与引用\\
int x=3; const int \& y=x;\\
//x=10;  \verb|---->|正确;\\
//y=20;  \verb|---->|错误:因为此时为常引用\\
\end{flushleft}


\section{C++函数新特性:缺省、重载、内联}
\begin{flushleft}
1、函数参数的默认值(缺省值):\\
有默认参数值的参数必须在参数表的最右端\\
\quad void fun(int i,int j=8):正确\\
\quad void fun(int i,int j=5,int k);错误\\
Tips:声明的时候把缺省值写上,用函数的时候不用写缺省值。\\
2、函数重载(名称相同参数可辨)\\
(1)定义:在相同作用域内,用同一函数名定义的多个函数,其中,参数个数和参数类型可以不同\\
3、内联函数(关键字:inline)\\
(1)定义:编译时,将函数体代码和实参代替函数调用语句\\
(2)声明的时候加inlin,使用时和普通函数调用一样:\\
inline int max(int a,int b,int c)\\
(3)限制:必须逻辑简单,不能用for、while循环,递归等等复杂结构
\end{flushleft} 

\section{内存管理}
\begin{flushleft}
1、定义:申请/归还内存资源就是内存管理\\
2、申请内存用new,释放内存用delete,这俩都是运算符不是函数\\
3、方法:\\
\quad (1)申请:\\
\qquad int *p=new int(10);\verb|-->|单个的 \\
\qquad int *arr=new int[10];\verb|-->|块内存\\
\quad (2)释放:\\
\qquad delete p \verb|-->|单个释放;\\
\qquad delete [ ]arr; \verb|-->|块释放;\\
4、注意事项\\
1、判断是否分配成功:\\
\qquad int *p=new int[1000]\\
\qquad if (NULL==p)\\
\qquad \{\\
\qquad    cout\verb|<<|"error!!"\\
\qquad \}\\
2、delete p; or delete []p之后要加一行p=NULL;


\part{封装篇(上)}
\begin{flushleft}
一、类和对象\\
 1、基本概念:数据成员,成员函数。\\
 2、访问限定符:public,protected,private.\\ 
 3、栗子:\\       
\quad class TV\\
\quad \{\\
\quad public:\\
\qquad char name[100];\\
\qquad  int type;\\

\qquad  void changevol();\\
\qquad  void power();\\
\quad private:\\
\qquad  电阻调节\\
\qquad  像素配色\\
\quad \}\\
4、成员对象的实例化\\
(1)从栈中实例化对象,与定义结构体没差:\\
\quad int main(void)\\
\quad \{\\
\qquad TV tv;\\
\qquad TV tv[100];\\
\qquad return 0;\\
\quad\}\\
(2)从堆中实例化对象:\\
\quad int main(void)\\
\quad \{\\
\qquad TV *p=new TV();\\
\qquad TV *q=new TV[100];\\
\qquad delete p;\\
\qquad delete [ ]q;\\
\qquad return 0;\\
\quad\}\\
5 成员对象的访问\\
(1)从栈中:\\
\qquad tv.type=0;\\
\qquad tv.changeVol();\\
(2)从堆中:\\
\qquad p\verb|->|type=4;\\
\qquad p\verb|->|changeVol();\\





\end{flushleft}





\end{flushleft}


 
\end{CJK*}
\end{document}